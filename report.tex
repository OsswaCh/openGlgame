\documentclass[a4paper, 12pt]{article}
\usepackage[utf8]{inputenc} % Encoding
\usepackage{graphicx} % For including images
\usepackage{titlesec} % For customizing section titles
\usepackage{geometry} % For adjusting page margins
\usepackage{hyperref} % For hyperlinks

\geometry{
    top=2.5cm,
    bottom=2.5cm,
    left=2.5cm,
    right=2.5cm,
    headheight=15pt,
    includeheadfoot
}

\pagestyle{fancy}
\fancyhf{}
\rhead{\thepage}
\lhead{\textit{OpenGLGame Report}}
\renewcommand{\headrulewidth}{1pt}

\titleformat{\section}{\normalfont\Large\bfseries}{\thesection}{1em}{}
\title{OpenGLGame Report}

\begin{document}
\maketitle
\tableofcontents
\newpage


\subtitle{Headers}
assert.h is a header file in the C programming language that provides the assert macro used for debugging and error checking during program development. It's used to verify assumptions made in the program's logic and halt the program's execution if an assertion fails.
windows.h is a header file in the Microsoft Windows operating system environment that contains declarations, definitions, and function prototypes for the Windows API (Application Programming Interface).

\subtitle{Notes about the code}

PeekMessage(&msg, NULL, NULL, NULL, PM_REMOVE)
PeekMessage checks the message queue for a pending message that corresponds to the specified criteria.
The PM_REMOVE flag indicates that the message should be removed from the queue after being retrieved.
If a message exists in the queue, PeekMessage returns a non-zero value (true), otherwise, it returns zero (false).
If a message is found and removed (PM_REMOVE), it's stored in the msg structure.

TranslateMessage(&msg);
is specifically designed to handle messages related to keyboard input, converting virtual-key messages into character messages

DispatchMessage(&msg);
dispatches the retrieved message to the appropriate window procedure for further processing.
The window procedure is responsible for handling various types of messages, such as user input, system notifications, and other events, and taking appropriate actions based on the message received.
